\section{\texttt{While} language} % approx. 2 pages
% Grammar
% Explanation of language
% Reference to the paper it was essentially taken from

For our target language,
we choose a simple While-language (shown in fig. \ref{fig:whilegrammar}).
\begin{figure}[H]
    \begin{lstlisting}
program ::= (v+) ";" statements
statements ::= (s ";")+

s ::= "skip"
    | "assert" f
    | v "<-" e
    | "if" c "then" statements "else" statements "end"
    | "while" c "invariant" f "do" statements "end"

e ::= [integer]
    | e [op] e

c ::= c "&&" c
    | c "||" c
    | e [cmp] e

f ::= "true"
    | "false"
    | f "\/" f
    | f "/\" f
    | f "->" f
    | "exists" v "." f
    | "forall" v "." f
    | e [cmp] e \end{lstlisting}
    \caption{Grammar for a simple While-language\label{fig:whilegrammar}}
\end{figure}

For the sake of simplicity,
all variables must be specified at the start of program.
Expressions $e$ are arithmetic integer expressions
supporting addition, subtraction, multiplication, division, and remainders.
Conditions $c$ may consist of booleans, \inl{\&\&}, \inl{||}, and integer comparisons.
Formulas $f$ are a small set of logical formulas, covering all properties that conditions can test
as well as implications and quantifications.

We borrow this language from a subset of \cite{jlamp},
although the only differences from a typical While-language
are the additions of logical assertions and loop invariants.
This is enough to write some simple programs and prove a few properties.
