\section{Implementation} % approx. ? pages. i think i added this point myself.

% Okay, maybe rather than a tutorial,
% I should think of it as writing documentation for the Why3 API.
% Then write about my concrete implementation here.
% My module can then serve as a template for more complicated languages.

% See, that's actually a pretty decent idea.
In this project,
we build a plugin that translates our While language directly into the base language.
This allows us to explore the viability of building custom VC-generators for new languages.

To ease the process, we use the Dune build system\cite{dunesite} and require Why3 as a dependency.
Additional dependencies are embedded (statically linked) in the executable
using the\\ % TODO: check this newline
\inl{embed\_in\_plugin\_libraries} property.

\subsection{Syntax tree}

For the sake of readability,
we define a our AST with zero Why3 dependencies.
\lstinputlisting[lastline=32]{../plugin/lib/wi_ast.ml}
Nodes in this tree are recursively tagged with position information,
with the intent of carrying these annotations through the language conversion.
The only thing of note is the sequence statement,
which holds a list of statements rather than two.

\subsection{Parsing}

For parsing, we use the MParser library\cite{mparser}.
It is a monadic parser combinator library with several useful features.
It lets us avoid the more complex build process
that is necessary in order to use parser generators
(although Dune does seem to have some builtin support for Menhir).
It has official support for position information,
which is lacking in other combinator libraries.
Finally, it has a built-in abstraction for parsing expressions with both prefix and infix operators,
operator precedence, and left/right associativity.

\subsection{VC-generation}

As we register our own AST as a child of the base language,
we can (and must) perform VC-generation when converting our AST.
Here we implement the weakest precondition calculus described in section \ref{sec:whilewp}.

In order to support integer arithmetic and comparisons,
we must import the theories \inl{int.Int} and \inl{int.ComputerDivision},
as the base theory only supports equality.
These are provided by the environment to our parser at runtime,
so we must carry the environment as part of our language.
