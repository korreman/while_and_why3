\section{Weakest precondition calculus} % approx. 2 pages
% An explanation of Weakest Precondition theory
% The WP for our while language

In order to generate verification conditions for a language,
we use an instance of Dijkstra's \textit{weakest precondition calculus}.
The WP calculus views statement as predicate transformers,
namely transforming from a set of postconditions to preconditions.

\newcommand{\wpp}{\mathrm{wp}}
\renewcommand{\implies}{\rightarrow}

Given a statement $S$, a precondition $P$, and a postcondition $Q$,
the weakest precondition $R$ is a predicate defined
such that $\{P\} S \{Q\}$ if and only if $P \implies R$.
In other words, it is the least strict condition that must be satisfied
in order to show $\{P\} S \{Q\}$.
We denote the weakest precondition by $\wpp(S, Q)$.

Given a program $B = S_1,~S_2,~...~S_n$, a WP-calculus $\wpp$, and a set of proof goals $G$,
we can obtain a set of verification conditions by applying our WP-calculus in reverse order:
\[
    VC(\wpp, B, G) = wp(S_1, \wpp(S_2, (... wp(S_n, G)))
\]

\subsection{WP-calculus for While} \label{sec:whilewp}

We roughly borrow the WP-calculus from \cite{jlamp} for use in our project.
\begin{align}
    \wpp(\inl{skip}, Q) &= Q \\
    \wpp(S_1; S_2, Q) &= \wpp(S_1, \wpp(S_2, Q)) \\
    \wpp(\inl{assert } P, Q) &= P \land Q ~~~~~~~~~~~~~~~~~~~~~~~~~~~~(\text{asymmetric})\\
    \wpp(x := e, Q) &= \forall v.~v = e \implies Q[x \leftarrow v] \\
    \wpp(\inl{if } c \inl{ then } S_1 \inl{ else } S_2, Q) &=
        \inl{if } c \inl{ then } \wpp(S_1, Q) \inl{ else } \wpp(S_2, Q) \\
    \wpp(\inl{while } e \inl{ invariant } I \inl{ do } s \inl{ end}, Q) &=\\
        I \land \forall \overset{\rightarrow} v. ~
        (I \implies \
        \inl{if } e \inl{ then } &\wpp(s, I) \inl{ else } Q)
        [\overset{\rightarrow} w \leftarrow \overset{\rightarrow} v]
\end{align}

The reasoning behind \inl{skip} and sequencing are fairly self-explanatory.

% Asymmetric conjunction
Assertions translate to an asymmetric conjunction.
While semantically equivalent to the usual conjunction, it communicates to Why3
that a goal $a \land b$ can be split into separate goals $a$ and $a \implies b$.

% Assignment
For assignments, performing a substitution $Q[x \leftarrow e]$ would be semantically sufficient.
However, by binding $e$ to a new variable $v$ and substituting with this instead,
we ensure that some large expression $e$ is not duplicated for every occurrence of $x$ in $Q$.
This prevents our verification conditions from growing exponentially with program size.

% if-else
Our \inl{if-else} expression is transformed into a \textit{logical} if-else formula.
It is equivalent to:
\[
    (c  \rightarrow \wpp(S_1, Q)) ~~~\land~~~
    (\neg c  \rightarrow \wpp(S_2, Q))
\]

% while
For \inl{while} statements,
$\overset{\rightarrow} w$ denotes all variables that may be modified in the loop,
while $\overset{\rightarrow} v$ denotes an equal amount of fresh variables.
The weakest precondition can be broken down as follows:
\begin{itemize}
    \item Base case: invariant $I$ must hold before entering the loop.
    \item Inductive step: for all values that may be modified in the loop,
    \begin{itemize}
        \item If the condition $c$ is true, the invariant $I$ must hold after running the loop body.
        \item Otherwise, the postcondition $Q$ must hold.
    \end{itemize}
\end{itemize}
