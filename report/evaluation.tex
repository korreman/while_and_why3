\section{Evaluation} % approx. 2 pages

The final product is a clear and well-documented example of a Why3 plugin.
Regrettably, the plugin is not thoroughly tested besides a few simple examples,
so it serves as a demonstration more than anything else.
The greatest limitation of the language and plugin in terms of features
is the lack of functions and predicates.
Another feature that wasn't implemented within the bounds of this project was position annotations,
although the plumbing for this feature is already laid out.
With testing and a few additional features,
the plugin could serve as a great baseline
to which experimental features and their VC generators could be explored.

In this project, we made the choice to convert our AST directly to MLW.
In fact, we could've chosen to register a format parser for MLW
rather than building our own syntax tree.
It might have been simpler and easier to convert our language to MLW instead.
There is nothing in our simple language
that doesn't have a parallel in the rather rich features of WhyML.
However, the API for building MLW syntax trees isn't necessarily simple either.
For a relatively simple language,
implementing a VC-generator is a manageable task.
Choosing to write a conversion also results in a clearer separation of concerns,
making the plugin a good starting point as a template for more advanced plugins.
