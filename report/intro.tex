\section{Introduction} % approx. .5 page
% So, adjusted this is 3 pages.
% What should I be talking about here?
% I'm out of practice when it comes to way-too-long reports about relatively simple work.

% Explanation of what Why3 really is.
Why3\cite{why3} can be best described as a framework for deductive program verification.
It provides the language WhyML for specification and programming,
using modules written in this language, it can:
\begin{itemize}
    \item Generate and discharge verification conditions to external theorem provers,
        namely SMT-solvers and proof assistants.
    \item Transform proof obligations.
    \item Generate program code in a number of target language.
\end{itemize}

Using WhyML as a compilation/transpilation target,
Why3 is able to function as an intermediate representation layer
between several languages and provers.
It also provides a GUI editor.
Using this editor, proof goals can be transformed and dispatched selectively to different solvers.
This allows for a semi-automated approach to proving,
with goal splitting, introspection, and manual proving when needed.

Aside writing in or converting to WhyML,
it is also possible to plug into the Why3 framework using plugins.
Using plugins, it is possible to define new language formats,
new languages with unique verification condition generators,
and new transformations of proof goals.

In this project, we implement a Why3 plugin for a small toy language
with its own VC-generator.
